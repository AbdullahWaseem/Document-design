% Three non-trivial examples to use algorithms in Latex.
% It features branches, loops and comments.


\documentclass[aspectratio=169]{beamer}

\usepackage{algorithm}
\usepackage{algpseudocode}

% Redefine \Return to have a new line before it (http://tex.stackexchange.com/questions/238574/algorithm-return-statement-does-not-begin-on-new-line)
\let\oldReturn\Return
\renewcommand{\Return}{\State\oldReturn}
\newcommand{\vars}{\texttt} % for variables to be more visible
\algnewcommand{\LineComment}[1]{\State \(\triangleright\) #1} % for a comment stretching a whole line (http://tex.stackexchange.com/questions/74880/algorithmicx-package-comments-on-a-single-line)
\algnewcommand\AND{\textbf{AND}} % (http://tex.stackexchange.com/questions/50908/algorithms-and-boolean-operator-casuing-undefined-control-sequence-error)
\algnewcommand\NOT{\textbf{NOT}}
\algnewcommand\ALL{\textbf{ALL}}
\algnewcommand\ANY{\textbf{ANY}}


% Only for the last example:

\DeclareMathOperator*{\argmax}{arg\,max}
\newcommand{\st}{\ |\ } % such that in set-builder notation


\begin{document}

\begin{frame}
   \begin{algorithm}[H]
       \caption{Domain integration on cut element by triangulation}
       \begin{algorithmic}[1]
           \State Given: \vars{grainPart}, \vars{method='subdivision'}, \vars{quadratureOrder}
           \State \vars{cutDomain} $\gets$ find the cut domain which belongs to \vars{grainPart}
           \ForAll {connected components of \vars{cutDomain}}
               \State Triangulate the connected set (e.g. CDT)
               \ForAll {triangles}
                   \State Calculate (constant) Jacobian $J_{\mathrm{q}}$
                  \ForAll {quadrature point $x_{\textrm{qp}}$ in triangle}
                     \State $x = \chi_2(x_{\textrm{qp}})$ \Comment{map from integration domain to physical domain}
                     \State $x_{\textrm{ref}} = \chi^{-1}(x)$ \Comment{map from physical domain to reference domain}
                     \State $\nabla_xf(x_{\textrm{ref}}) = \mathbf{J}^{-1}(x_{\textrm{ref}})\cdot\nabla_{x_{\mathrm{ref}}}f(x_{\textrm{ref}})$ \Comment{gradient in physical coord.}
                     \State Form $B(x_{\textrm{ref}})$ from $\nabla_xf(x_{\textrm{ref}})$ \Comment{strain-displacement matrix}
                     \State Update $\mathbf{B}^\top \mathbf{C} \mathbf{B} \det J_{\mathrm{q}} w_{\textrm{qp}}$ \Comment{additivity of integral}
                  \EndFor
               \EndFor
           \EndFor
       \end{algorithmic}
   \label{alg:triangulation}
   \end{algorithm}
\end{frame}



\begin{frame}
    \frametitle{\insertsection\ -- \insertsubsection}
    \begin{algorithm}[H]
        \caption{Monolithic creation of the stiffness matrix}
        {\fontsize{11}{11}\selectfont
        \begin{algorithmic}[1]
            \State Given: storeRequest (true/false) \Comment{store locally assembled matrices/vectors}
            \ForAll {subdomains}
                \State Determine local (subdomain) $\rightarrow$ global mapping
                \ForAll {grains}
                   \ForAll {elements}
                      \State Calculate $K^{\mathrm{elem}}$
                      \State Determine element $\rightarrow$ local (subdomain) mapping
                      \If {storeRequest}
                         \State Assemble $K^{\mathrm{elem}}$ to $K^{\mathrm{local}}$
                         \State Store $K^{\mathrm{local}}$
                      \EndIf
                      \State Assemble $K^{\mathrm{elem}}$ to $K^{\mathrm{global}}$
                   \EndFor
                \EndFor
            \EndFor
        \end{algorithmic}
        }
        \label{alg:monolithicStiffness}
    \end{algorithm}
\end{frame}



\begin{frame}
        \frametitle{\insertsubsection}
       \begin{itemize}
           \setlength\itemsep{0em}
           \item \only<1>{\footnotesize {Original algorithm}} \only<2>{\alert<2>{\footnotesize {Modified algorithm}}}
           \vspace{-3mm}
           \begin{algorithm}[H]
               \caption{Find elements with local maximum stress}
               \footnotesize{
               \begin{algorithmic}[1]
                   \State Given: element set $E$, element connectivity $N$\only<2>{, \alert{cut elements $C$}}
                   \State $S = \{e\in E \st \only<1>{\sigma_1} \only<2>{\alert{\sigma_e}} > \sigma_c \}$ \Comment{possibly failing elements\only<2>{, \alert{other stress criterion}}}
                   \While {$S \neq \emptyset$} \Comment{until all admissible elements are placed into a cluster}
                      \State $e = \argmax_{e\in S}\only<1>{\sigma_1(S)}\only<2>{\alert{\sigma_e(S)}}$ \Comment{element with locally largest stress value}
                      \State $Q = \{ e \}$ \Comment{admissible elements in the current layer}
                      \State $S_n = \{ e \}$ \Comment{container for admissible elements in a cluster}
                      \While {$Q \neq \emptyset$} \Comment{until there are admissible elements locally}
                         \State $Q = \{ e\in N(Q) \only<2>{\alert{\wedge e\in C}}\st \sigma_c \leq \sigma(e) \leq \only<1>{\sigma_1(Q)} \only<2>{\alert{\sigma_e(Q)}} \}$ \Comment{criterion for neighbouring elements}
                         \State $S_n = S_n \cup Q$ \Comment{accumulate elements in the local cluster}
                      \EndWhile
                         \State $S = S \setminus S_n$ \Comment{remove cluster elements from the whole}
                   \EndWhile
               \end{algorithmic}
               }
               \label{alg:stressCriterion}
           \end{algorithm}
       \vspace{-3mm}
       \item \footnotesize{Create the clusters $S_n$ (including the elements in $\sigma_1$-decreasing order) of unconnected element sets where the stress criterion is locally fulfilled}
       \item \footnotesize{Breadth-first search on the element connectivity graph}
       \end{itemize}
    \end{frame}

\end{document}


