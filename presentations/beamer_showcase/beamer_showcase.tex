% TODO
% Example of a TikZ diagram with overlays
% Fix texshade features font color set to black (issue with dark background)
% Fix some lines in circuitikz set to black (issue with dark background)
% More comments (which?)

\RequirePackage[l2tabu,orthodox]{nag} % good practice

\documentclass[aspectratio=169% adapt to your display device (e.g. 1610, 43)
]{beamer}

\usepackage[british]{babel}
\usepackage{csquotes}
\usepackage{microtype}
\usepackage{lmodern} % for any symbol missing in typefaces load afterwards
\usepackage{amsmath,amssymb} % many math-related features
\usepackage{subcaption} % subfigures with their own captions
\usepackage{graphicx} 
\usepackage{pgfplots}
\usepackage{tikzscale} % (it loads tikz)
\usepackage{hyperref}
\usepackage{booktabs} % for nicely formatted tables
\usepackage{siunitx} % for units and quantities
\usepackage{cleveref}
\usepackage{xfrac} % for \sfrac
\usepackage{mathtools} % for multlined environment
\usepackage{appendixnumberbeamer} % for appendix (backup) frames

\usepackage[version=3]{mhchem}   % sufficient for chemical equations
\usepackage{chemfig}   % for 2-D molecule drawings
\usepackage{texshade} % for nucleotide and peptide alignments
\usepackage[siunitx]{circuitikz} % for circuit diagrams
\usepackage{smartdiagram} % predefined organisational diagrams

\renewcommand{\printatom}[1]{\ensuremath{\mathsf{#1}}} % sans serif font in chemical figures
\usepgfplotslibrary{statistics} % for box plots

% Version with biblatex
\usepackage[style=authoryear, % try others, e.g. 'numeric' (see biblatex manual)
url=false % no url in bibliography
]{biblatex}
\addbibresource{references.bib}
% For a version without biblatex:
% 1. comment the two previous lines,
% 2. follow instruction on frame 'Citations', and
% 3. follow instruction on frame 'References'.

% Theme
\usetheme[%background=dark, % uncomment to 'invert' colours
block=fill, % see metropolis manual for this option and the following ones
sectionpage=progressbar,
subsectionpage=progressbar,
progressbar=frametitle]{metropolis}
% Set progress bar width (requires a metropolis version >=1.2)
\makeatletter 
\setlength{\metropolis@progressinheadfoot@linewidth}{1pt}
%\setlength{\metropolis@titleseparator@linewidth}{1pt}
%\setlength{\metropolis@progressonsectionpage@linewidth}{1pt}
\makeatother
% Alternative colour theme
% \usecolortheme{solarizeddark}


% Alternative theme
% If you uncomment the following lines, comment the whole 'Theme' block above, and every line with the comment 'metropolis-only' in the document.
% \usetheme{CambridgeUS} % or Madrid, Frankfurt
% \usecolortheme{dolphin} % or solarizeddark, as suggested above (but load the theme first)
% \AtBeginSection{ % Remind progress through table of content at each section
% \begin{frame}[noframenumbering]{Outline}
%   \tableofcontents[currentsection]
% \end{frame}
% }
% Themes and colour themes built in beamer (i.e. requiring no extra package) can be found at https://mpetroff.net/files/beamer-theme-matrix/

\beamertemplatenavigationsymbolsempty % remove navigation symbols
\setbeamercovered{transparent} % covered objects are transparent (instead of, e.g., invisible)

% Typefaces (uncomment only if compiled with pdfLaTeX)
\usepackage[T1]{fontenc}
\usepackage{lmodern}
\usepackage{eulervm} % for math (comment if you prefer Latin Modern Math)
% Uncomment only one of the following lines
\usepackage[sfdefault,light]{FiraSans} % recommended typeface for metropolis theme. Comment if you compile with LuaLaTeX since metropolis then loads it itself (see metropolis manual).
\usepackage{FiraMono} % monospaced font (comment to use Latin Modern Mono)
% Suggestions of other text typefaces
% \usepackage[math]{iwona}
% \usepackage[math]{kurier}
% \usepackage{libertine}\usepackage{libertinust1math}

% Captions (remove labels)
\captionsetup{labelformat=empty}
\captionsetup[subfigure]{labelformat=empty}

% siunitx setup
\sisetup{retain-unity-mantissa=false,
  exponent-product=\cdotp, % or \cdotp
  binary-units=true,
  detect-all,
  list-final-separator={ \sffamily and } % fix to set list separator in sans serif family
}
% Fix for siunitx: it expects \lseries to be defined for a light-weight font
\makeatletter
\DeclareRobustCommand\lseries{\not@math@alphabet\mdseries\relax\fontseries{l}\selectfont}
\makeatother

% Title page data
\titlegraphic{\includegraphics[width=2cm]{logo_gem}}
\title[My short title]{My long title}
\author[F.\,N. Surname ]{First Name \textsc{Surname}}
\institute[Short affiliation]{My long affiliation}
\date[2018-07-05]{Beamer showcase, 5\textsuperscript{th} July 2018, Nantes, France}

\begin{document}

\maketitle{}

\frame{{Outline}\tableofcontents}

% \begin{frame}
%   \frametitle{Outline}
%   \tableofcontents{}
% \end{frame}

\section{Formatting text}

\begin{frame}{Paragraphs}{Not such a useful slide}
  Suspendisse vel felis. Ut lorem lorem, interdum eu, tincidunt sit amet, laoreet
  vitae, arcu. Aenean faucibus pede eu ante. Praesent enim elit, rutrum at, molestie
  non, nonummy vel, nisl. Ut lectus eros, malesuada sit amet, fermentum eu,
  sodales cursus, magna. Donec eu purus. Quisque vehicula, urna sed ultricies
  auctor, pede lorem egestas dui, et convallis elit erat sed nulla. Donec luctus.
  Curabitur et nunc. 

  Morbi luctus, wisi viverra faucibus pretium, nibh est placerat odio, nec commodo
  wisi enim eget quam. Quisque libero justo, consectetuer a, feugiat vitae, porttitor
  eu, libero. Suspendisse sed mauris vitae elit sollicitudin malesuada. Maecenas
  ultricies eros sit amet ante. Ut venenatis velit. Maecenas sed mi eget dui varius
  euismod. Phasellus aliquet volutpat odio. Vestibulum ante ipsum primis in
  faucibus orci luctus et ultrices posuere cubilia Curae; Pellentesque sit amet pede
  ac sem eleifend consectetuer. 
\end{frame}

\begin{frame}{Typography}
  Besides \emph{emphasised} and \textbf{bold} text, you can use coloured \alert{accents}.
\end{frame}

\begin{frame}{Default blocks}
  Three block environments are always defined by the theme.
  
  \begin{block}{1\textsuperscript{st} block}
    \texttt{block} environment.
  \end{block}
  
  \begin{exampleblock}{2\textsuperscript{nd} block}
    \texttt{exampleblock} environment.
  \end{exampleblock}
  
  \begin{alertblock}{3\textsuperscript{rd} block}
    \texttt{alertblock} environment.
  \end{alertblock}

  Obviously, you can define your own.  
\end{frame}


\begin{frame}{Lists and columns}
  \begin{columns}[T,onlytextwidth]
    \column{0.33\textwidth}
    Items
    \begin{itemize}
    \item Milk
    \item Eggs
    \item Potatoes
    \end{itemize}

    \column{0.33\textwidth}
    Enumerations
    \begin{enumerate}
    \item First,
    \item Second and
    \item Last.
    \end{enumerate}

    \column{0.33\textwidth}
    Descriptions
    \begin{description}
    \item[PowerPoint] Meeh.
    \item[Beamer] Yeeeha.
    \end{description}
  \end{columns}
\end{frame}

\begin{frame}{Some math}
  \begin{columns}[onlytextwidth]
    \column{.45\textwidth}
    \begin{align}
      \exp(\varpi) & = \sum_{x=0}^{+\infty}\frac{\varpi^{x}}{\varpi!} \\
      \label{eq:exp-expansion}
                   &
                     \begin{multlined}[b]
                       = 1 + \varpi + \frac{\varpi^{2}}{2} + \frac{\varpi^{3}}{6} \\
                       + \frac{\varpi^{4}}{24}+ o_{0}(\varpi^{5})          
                     \end{multlined}
    \end{align}
    \column{.5\textwidth}
    The number \(e\) is transcendental yet, from \cref{eq:exp-expansion} and since \(e=\exp(1)\), we see that a reasonable approximation could be \(e\approx 2 + \sfrac{1}{2} + \sfrac{1}{6} + \sfrac{1}{24} \approx 2.7083\). Indeed, \(e\approx2.7183\).
  \end{columns}
\end{frame}

\begin{frame}{Citations}
  Various types of citations\footnote{Depends on the citation style.}:
  \cite{Knuth92}, \footcite{ConcreteMath}, \textcite{Simpson}, \footfullcite{Er01}, \parencite{greenwade93}
  % Version without biblatex:
  % \cite{Knuth92}, \footnote{\cite{ConcreteMath}}
\end{frame}

\section{Visual content}
\label{sec:content}

\subsection{\enquote{Floats}}

\begin{frame}{Tables}
  \centering{}
  \begin{table}
    \centering{}
    \caption{Meat prices (source: Wikipedia)}
    \begin{tabular}{llr}
      \toprule
      \multicolumn{2}{c}{Item} &                          \\
      \cmidrule{1-2}
      Animal                   & Description & Price (\$) \\
      \midrule
      Gnat                     & per gram    & 13.65      \\
                               & each        & 0.01       \\
      Gnu                      & stuffed     & 92.50      \\
      Emu                      & stuffed     & 33.33      \\
      Armadillo                & frozen      & 8.99       \\
      \bottomrule
    \end{tabular}
  \end{table}
\end{frame}

\begin{frame}{Figure with caption}
  \centering{}
  \begin{figure}
    \centering
    \begin{tikzpicture}
      % \draw[help lines,gray!20] (0,0) grid (6,4);
      % \draw[semithick] (0,0) rectangle (6,4);
      \draw[semithick,->,>=latex'] (-.5, 0) -- (6.5, 0);
      \draw[semithick,->,>=latex'] (0, -.5) -- (0,5);
      \draw[dashed,semithick] (0, 0.2) parabola (3.5,4) node[above,font=\footnotesize]{impossible to do} (3.5, 4);
      \node[left] at (3,3) {Word\textsuperscript{\textregistered}};
      \draw[semithick] (0, 1) parabola (6,3); 
      \node at (5.2,2) {\LaTeX};
      \node[below,font=\footnotesize] at (3,0) {document complexity and size};
      \node[xshift=-8pt,yshift=4pt,font=\footnotesize,transform shape,rotate=90] at (0,2.2) {effort and time consumption};
    \end{tikzpicture}
    \caption{Scalability of \LaTeX\ and Microsoft Word\textsuperscript{\textregistered} against document size and complexity (redrawn from Marko Pinteric's original at \url{http://www.pinteric.com/miktex.html})}
  \end{figure}
\end{frame}

\begin{frame}{Subfigures with subcaptions}
  \centering{}
  \begin{figure}
    \centering{}
    \begin{subfigure}[c]{.6\textwidth}
      \centering{}
      \begin{tikzpicture}
        \draw[rounded corners] (0,0) rectangle (4,4) node at (2,2) {1\textsuperscript{st} figure};
      \end{tikzpicture}
      \caption{1\textsuperscript{st} subcaption}
    \end{subfigure}
    ~
    \begin{subfigure}[c]{.3\textwidth}
      \centering{}
      \begin{tikzpicture}
        \draw[rounded corners] (0,0) rectangle (2,2) node at (1,1) {2\textsuperscript{nd} figure};
      \end{tikzpicture}
      \caption{2\textsuperscript{nd} subcaption}
    \end{subfigure}
    \caption{This is a caption}
  \end{figure}
\end{frame}

\subsection{Charts and diagrams}

\begin{frame}{Line plots}
  \centering
  \begin{tikzpicture}
    \begin{axis}[
      xmajorgrids,
      ymajorgrids,
      major grid style={dotted},
      axis x line=bottom,
      axis y line=left,
      width=0.9\textwidth,
      height=6cm,
      mlineplot, % metropolis-only (uncomment only with metropolis theme)
      ]
      \addplot {sin(deg(x))};
      \addplot+[samples=100] {sin(deg(2*x))};
    \end{axis}
  \end{tikzpicture}
\end{frame}

\begin{frame}{Bar charts}
  \centering
  \begin{tikzpicture}
    \begin{axis}[
      ymajorgrids,
      major grid style={dotted},
      axis x line*=bottom,
      axis y line*=left,
      ybar,
      area legend,
      xlabel={Foo},
      ylabel={Bar},
      width=0.9\textwidth,
      height=6cm,
      legend style={draw=none,fill=none},
      mbarplot, % metropolis-only (uncomment only with metropolis theme)
      ]
      \addplot plot coordinates {(1, 20) (2, 25) (3, 22.4) (4, 12.4)};
      \addplot plot coordinates {(1, 18) (2, 24) (3, 23.5) (4, 13.2)};
      \addplot plot coordinates {(1, 10) (2, 19) (3, 25) (4, 15.2)};
      \legend{lorem, ipsum, dolor}
    \end{axis}
  \end{tikzpicture}
\end{frame}

\begin{frame}{Box plots}
  \centering{}
  % With tikzscale package
  \includegraphics[width=\linewidth,height=.7\paperheight]{box-plots.tikz}
  % Basic version
  % \begin{tikzpicture}
  \begin{axis}
    [
    ytick={1,2,3},
    yticklabels={Index 0, Index 1, Index 2},
    ]
    \addplot+[
    boxplot prepared={
      lower whisker=2.214844,
      lower quartile=3.608312,
      median=3.895478,
      upper quartile=4.447298,
      upper whisker=4.666284
    }
    ]
    coordinates{
      (0,4.832228)
      (0,5.513942)
      (0,6.29165)
      (0,5.216712)
      (0,5.677036)
      (0,4.981995)
      (0,5.172095)
      (0,4.056417)
    };
    \addplot+[
    boxplot prepared={
      lower whisker=3.508799,
      lower quartile=5.079821,
      median=5.481519,
      upper quartile=5.971588,
      upper whisker=6.250831
    },
    ] coordinates{
      (0,6.508115)
      (0,6.486354)
      (0,6.860059)
      (0,6.620663)
      (0,7.312391)
      (0,7.357306)
      (0,6.421694)
      (0,6.479597)
      (0,6.690945)
      (0,6.661593)
      (0,7.271025)
      (0,6.396931)
      (0,7.035161)
      (0,7.371248)
      (0,7.033689)
      (0,7.002645)
      (0,6.590617)
      (0,7.171933)
      (0,6.416259)
      (0,5.552438)
    };
    \addplot+[
    boxplot prepared={
      lower whisker=2.437751,
      lower quartile=3.334956,
      median=4.336029,
      upper quartile=5.068459,
      upper whisker=5.265037
    }
    ]
    coordinates{
      (0,5.826492)
      (0,5.819791)
      (0,6.21436)
      (0,4.47166)
    };
  \end{axis}
\end{tikzpicture}
\end{frame}

\begin{frame}{Organisational diagrams}
  \begin{columns}[T]

    \begin{column}{.49\textwidth}
      \resizebox{\linewidth}{.75\paperheight}{\smartdiagram[bubble diagram]{Planning Cycle,Assess,Plan,Implement,Renew}}
    \end{column}

    \begin{column}{.49\textwidth}
      \resizebox{\linewidth}{.75\paperheight}{\smartdiagram[priority descriptive diagram]{Assess,Plan,Implement,Renew}}
    \end{column}

  \end{columns}

\end{frame}

\subsection{Field-specific visualisations}

\begin{frame}{Chemical Equations and Molecules}
  \centering\small
    \ce{Zn^2+ <=>[\ce{+ 2OH-}][\ce{+ 2H+}]
      $\underset{\text{amphoteres Hydroxid}}{\ce{Zn(OH)2 v}}$
      <=>C[\ce{+2OH-}][\ce{+ 2H+}]
      $\underset{\text{Hydroxozikat}}{\cf{[Zn(OH)4]^2-}}$
    }
    \hfil
    \chemfig{H-C(-[2]H)(-[6]H)-C(-[7]H)=[1]O}
\end{frame}

\begin{frame}{Life Sciences}
  \centering{}
  \vfill % helps to center vertically
  \begin{texshade}{AQPpro.MSF.txt}
    \alignment{center} % seems not to work
    % Font colour is set to black. Make it cleverer by setting to theme foreground colour
    \namescolor{normal text.fg}
    \numberingcolor{normal text.fg}
    \featurestylenamescolor{normal text.fg} % seems not to work
    \featurenamescolor{normal text.fg} % seems not to work
    % To reduce font size
    % \setsize{numbering}{small}
    % \setsize{names}{small}
    % \setsize{residues}{small}
    \setfont{features}{sf}{up}{md}{normalsize} % set 'features' font upright
    \shadingmode{similar} 
    \threshold[80]{50} 
    \setends{1}{80..112} 
    \hideconsensus % comment to show consensus
    \feature{top}{1}{93..93}{fill:$\downarrow$}{first case (see text)} 
    \feature{bottom}{1}{96..99}{brace}{second case (see text)} 
  \end{texshade}
\end{frame}


\begin{frame}{Circuits and SI Units}
  \begin{columns}
    \begin{column}{.49\textwidth}\rmfamily
      \begin{circuitikz}[transform shape,scale=1.2,font=\sffamily]
        \draw (0,0) node[anchor=east] {B}  to[short, o-*] (1,0)    to[R=20<\ohm>, *-*] (1,2)
        to[R=10<\ohm>, v=$v_x$] (3,2) -- (4,2)
        to[ cI=$\frac{\si{\siemens}}{5} v_x$, *-*] (4,0) -- (3,0)  to[R=5<\ohm>, *-*] (3,2)
        (3,0) -- (1,0)   (1,2) to[short, -o] (0,2) node[anchor=east]{A}
        ;\end{circuitikz}
    \end{column}
    \begin{column}{.49\textwidth}
      \begin{itemize}\rmfamily
      \item \SI{3.45d4}{\square\volt\cubic\lumen\per\farad}
      \item \SIlist[per-mode=symbol]{40;85;103}{\kilo\metre\per\hour}
      \end{itemize}
    \end{column}
  \end{columns}
\end{frame}

\section{Overlays}

\begin{frame}{Simple overlays}
  \onslide*<1-3,5>{
    \begin{itemize}
    \item<+-> First item;
    \item<+-> second item;
    \item<+-> \dots;
    \item<5-> last item.
    \end{itemize}
  }
  \onslide*<4>{
    \centering{}
    \begin{tikzpicture}
      \path[fill=white,draw=cyan] (2,2) circle (2) node at (2,2) {\color{blue} Awesome picture} ;
    \end{tikzpicture}
  }
\end{frame}

\begin{frame}{Intricate overlays}
  \begin{enumerate}
  \item<+-> This is\onslide*<2->{ \alert<5->{\textbf<4->{\emph<3->{really}}}} important
  \item<+-> This \temporal<.(1)>{will change}{is changing}{has changed}
  \item<+> This item will be covered again
  \onslide*<1-+>{\item<.-> This item will be removed}
  \item<.-> \onslide+<.-+>{This text will become invisible}
  \end{enumerate}
  This demonstration is \alt<.(1)-.(2)>{over}{going on}. \pause The end.
\end{frame}

\begin{frame}{Clever overlays}
  \centering
  \includegraphics<+,+(5)>[height=.9\textheight]{lenna}
  \begin{itemize}[<+-5 | alert@+>]
  \item \alert<.(3)>{You are looking at this\only<.(3)>{ again!}}
  \item Now at this
  \item And now at this
  \end{itemize}

  \onslide+<5> For a detailed explanation of overlays, see \texttt{beamer}'s user guide, § 9.
\end{frame}

\section{Conclusion}

{% Make this frame stand out (only without metropolis)
  % \setbeamercolor{background canvas}{bg=structure.fg}
  % \setbeamercolor{normal text}{fg=structure.bg}
  % \usebeamercolor[fg]{normal text}
  % \setbeamertemplate{footline}{}
  % metropolis provides a frame option 'standout' to achieve this result (see its manual)
  \begin{frame}[standout] % metropolis-only (uncomment only with metropolis theme, otherwise uncomment lines above)
    \centering\huge\bfseries
    Any question?
  \end{frame}
}

\appendix

\begin{frame}[fragile]{Backup slides} % 'fragile' option necessary with \verb
  Sometimes, it is useful to add slides at the end of your presentation to refer to during audience questions.

  The best way to do this is to include the \verb|appendixnumberbeamer| package in your preamble and call \verb|\appendix| before your backup slides.
\end{frame}

\setbeamertemplate{frametitle continuation}{\insertcontinuationcountroman}
\setbeamertemplate{bibliography item}{\insertbiblabel}
\begin{frame}[allowframebreaks]{References}
  \nocite{*}
  % Version with biblatex
  \printbibliography[heading=none]
  % Version without biblatex (comment the two previous lines)
  % \bibliography{references}
  % \bibliographystyle{abbrv}
\end{frame}

\end{document}

%%% Local Variables:
%%% mode: latex
%%% TeX-master: t
%%% ispell-local-dictionary: "british"
%%% End: